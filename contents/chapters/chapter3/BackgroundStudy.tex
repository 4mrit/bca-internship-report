\section{Background Study}
In recent years, Software as a Service (SAAS) has transformed various industries, including healthcare. Traditional hospital management systems were often based on on-premise solutions that required significant investment in hardware and maintenance. With the advent of cloud computing, SAAS-based systems now provide a flexible, scalable, and cost-effective alternative.Recent studies have shown that cloud-based hospital management systems significantly improve operational efficiency \cite{smith2018cloud, doe2020digital}

Key concepts in this domain include:
\begin{itemize}
  \item \textbf{Cloud Computing:} A model that enables on-demand access to computing resources over the internet, allowing hospitals to scale services as needed.
  \item \textbf{Multi-Tenancy:} A software architecture where a single instance of the application serves multiple institutions while keeping their data isolated.
  \item \textbf{Virtualization:} The process of creating virtual versions of hardware or software, which underpins the efficient use of resources in cloud computing.
  \item \textbf{Scalability and High Availability:} Essential features that ensure the system can handle growth in data and user load while remaining reliable.
  \item \textbf{Service-Oriented Architecture (SOA):} An approach to design software where services communicate over a network, providing modularity and reusability.
\end{itemize}

These theories and concepts not only support the technical design of SAAS-based HMS but also help in addressing the growing challenges in modern healthcare such as cost reduction, improved accessibility, and enhanced patient care.

